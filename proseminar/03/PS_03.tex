\documentclass[UTF-8]{article}
\usepackage{amsmath}
\usepackage{amssymb}
\usepackage{float}
\usepackage{graphicx}
\usepackage{epstopdf}
\usepackage{inputenc}
\usepackage{geometry}
\usepackage{pgfplots} 
\usepackage{listings}
\usepackage{enumitem}
\usepackage{lipsum}  
\usepackage{color}
\usepackage[colorlinks=true, urlcolor=blue, linkcolor=blue]{hyperref}
\geometry{left=2.5cm,right=2.5cm,top=2.5cm,bottom=2.5cm}

\definecolor{codegreen}{rgb}{0,0.6,0}
\definecolor{codegray}{rgb}{0.2,0.2,0.2}
\definecolor{codepurple}{rgb}{0.58,0,0.82}
\definecolor{backcolour}{rgb}{0.95,0.95,0.952}

\lstdefinestyle{mystyle}{
	backgroundcolor=\color{backcolour},   
	commentstyle=\color{codegreen},
	keywordstyle=\color{blue},
	numberstyle=\tiny\color{codegray},
	stringstyle=\color{codepurple},
	basicstyle=\ttfamily\footnotesize,
	breakatwhitespace=false,         
	breaklines=true,                 
	captionpos=b,                    
	keepspaces=true,                 
	numbers=left,                    
	numbersep=5pt,                  
	showspaces=false,                
	showstringspaces=false,
	showtabs=false,                  
	tabsize=2,
	linewidth=1.185\linewidth,
	resetmargins=true,
	xleftmargin=-1cm,
	xrightmargin=0.085\textwidth,
	prebreak = \raisebox{0ex}[0ex][0ex]{\ensuremath{\hookleftarrow}}
}

\lstset{style=mystyle}


\title{High Performance Computing Proseminar 2024 \\
    \large Assignment 3} %exchange for assignment number

\author{Stefan Wagner \& Sebastian Bergner\\Team: Planning Desaster}
\begin{document}
    
    \maketitle
    
    The goal of this assignment is to extend the heat stencil application and measure its performance. 
    
    \section*{Exercise 1}
    This exercise consists in extending the heat stencil application of Assignment 2 to two dimensions. 
    
    \textbf{Tasks}
    
    
    \begin{itemize}
    	\item Extend the heat stencil application to the two-dimensional case and name it heat\_stencil\_2D.
    	\item Provide a sequential and an MPI implementation. Try to make your implementation as efficient as possible, also with regard to code
    	readability.
    	\item Run your programs with multiple problem and machine sizes and measure speedup and efficiency. Consider using strong scalability, weak
    	scalability, or both. Justify your choice.
    	\item Illustrate the data in appropriate figures and discuss them. What can you observe?
    	\item Measure and illustrate one domain-specific and one domain-inspecific performance metric. What can you observe?
    	\item How can you verify the correctness of your applications?
    	\item Insert the wall times for the sequential version and for 96 cores for N=768x768 and T=N*100 into the comparison spreadsheet: \href{https://docs.google.com/spreadsheets/d/1p6d9F12EtykmI2-7MnHkg0U15UAtaCvWz8Ip92ZEsWo}{Docs}
    \end{itemize}
    
    \section*{Exercise 2}
    
    This exercise consists in comparing blocking and non-blocking communication for the heat stencil applications. 
    
    \textbf{Tasks}
    
    \begin{itemize}
    	\item Provide an MPI implementation for the 1D and 2D heat stencil that uses non-blocking communication. If you already implemented a non-blocking version, provide a blocking version, but ensure the non-blocking version works as described below.
    	\item Structure your program such that you 1) start a non-blocking ghost cell exchange, 2) compute the inner cells which do not require the
    	result of the ghost cell exchange, 3) block until the ghost cell exchange has finished, and 4) compute the remaining cells.
    	\item Run your programs with multiple problem and machine sizes and compare both versions.
    	\item Insert wall time for 96 cores for N=768x768 and T=N*100 into the comparison spreadsheet: \href{https://docs.google.com/spreadsheets/d/1p6d9F12EtykmI2-7MnHkg0U15UAtaCvWz8Ip92ZEsWo}{Docs}
    \end{itemize}
\end{document}