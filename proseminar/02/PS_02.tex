\documentclass[UTF-8]{article}
\usepackage{amsmath}
\usepackage{amssymb}
\usepackage{float}
\usepackage{graphicx}
\usepackage{epstopdf}
\usepackage{inputenc}
\usepackage{geometry}
\usepackage{pgfplots} 
\usepackage{listings}
\usepackage{enumerate}
\usepackage{lipsum}  
\usepackage{color}
\usepackage[colorlinks=true, urlcolor=blue, linkcolor=blue]{hyperref}
\geometry{left=2.5cm,right=2.5cm,top=2.5cm,bottom=2.5cm}

\definecolor{codegreen}{rgb}{0,0.6,0}
\definecolor{codegray}{rgb}{0.5,0.5,0.5}
\definecolor{codepurple}{rgb}{0.58,0,0.82}
\definecolor{backcolour}{rgb}{0.9,0.9,0.92}

\lstdefinestyle{mystyle}{
    backgroundcolor=\color{backcolour},   
    commentstyle=\color{codegreen},
    keywordstyle=\color{blue},
    numberstyle=\tiny\color{codegray},
    stringstyle=\color{codepurple},
    basicstyle=\ttfamily\footnotesize,
    breakatwhitespace=false,         
    breaklines=true,                 
    captionpos=b,                    
    keepspaces=true,                 
    numbers=left,                    
    numbersep=5pt,                  
    showspaces=false,                
    showstringspaces=false,
    showtabs=false,                  
    tabsize=2
}

\lstset{style=mystyle}


\title{High Performance Computing Proseminar 2024 \\
    \large Assignment 2} %exchange for assignment number

\author{Stefan Wagner \& Sebastian Bergner}
\begin{document}
    
    \maketitle
    
    \section*{Exercise 1}
    This exercise consists in writing a parallel application to speed up the computation of $\pi$.
    
    There are many ways of approximating $\pi$, one being a well-known Monte Carlo method: The ratio of the areas of a square and its incircle is $\pi/4$. Since the exact area of a circle cannot be computed (we don't know the value of $\pi$ yet), one can instead sample random points, check their distance from the center and compute the ratio of points inside the circle to all sampled points.
    
    \begin{itemize}
    	\item Write a sequential application `pi\_seq` in C or C++ that computes $\pi$ for a given number of samples (command line argument). Test your application for various, large sample sizes to verify the correctness of your implementation.
    	\item Consider a parallelization strategy using MPI. Which communication pattern(s) would you choose and why?
    	\item Implement your chosen parallelization strategy as a second application `pi\_mpi`. Run it with varying numbers of ranks and sample sizes and verify its correctness by comparing the output to `pi\_seq`.
    	\item Discuss the effects and implications of your parallelization.
    	\item Insert the measured wall time for $10^9$ samples for the sequential implementation and on 96 cores for MPI into the  \href{https://docs.google.com/spreadsheets/d/1p6d9F12EtykmI2-7MnHkg0U15UAtaCvWz8Ip92ZEsWo}{comparison spreadsheet}
    \end{itemize}
    
    
    
    \section*{Exercise 2}
	This exercise consists in parallelizing an application simulating the propagation of heat.
	
	A large class of scientific applications are so-called stencil applications. These simulate time-dependent physical processes such as the propagation of heat or pressure in a given medium. The core of the simulation operates on a grid and updates each cell with information from its neighbor cells.
	
	
	\begin{itemize}
		\item A sequential implementation of a 1-D heat stencil is available in heat\_stencil\_1D\_seq.c. Read the code and make sure you understand what happens. See the Wikipedia article on \href{https://en.wikipedia.org/wiki/Stencil_code}{Stencil Codes} for more information.
		\item Consider a parallelization strategy using MPI. Which communication pattern(s) would you choose and why? Are there additional changes required in the code beyond calling MPI functions? If so, elaborate!
		\item Implement your chosen parallelization strategy as a second application `heat\_stencil\_1D\_mpi`. Run it with varying numbers of ranks and problem sizes and verify its correctness by comparing the output to `heat\_stencil\_1D\_seq`.
		\item Discuss the effects and implications of your parallelization.
		\item Insert the measured wall time for N=6144 and 96 cores into the  \href{https://docs.google.com/spreadsheets/d/1p6d9F12EtykmI2-7MnHkg0U15UAtaCvWz8Ip92ZEsWo}{comparison spreadsheet}
	\end{itemize}


\end{document}